% !TEX root = epifanov_solid_state_physics.tex
%!TEX TS-program = pdflatex
%!TEX encoding = UTF-8 Unicode


\chapter*{PREFACE}
\addcontentsline{toc}{chapter}{Preface}
\chaptermark{PREFACE}

\vspace*{-12pt}

Ten years have passed since the first Russian edition of this textbook was published. In this time solid state physics has developed rapidly as the scientific background of numerous front-line branches of technology, absorbing new discoveries and theories. This has been considered in preparing the new edition.

At the same time college curricula have been changed to improve the basic preparation of versatile engineers, especially in physics and mathematics. This too had to be reflected in this book.

Also, the years that have elapsed since the first edition have seen much comment, some critical, and many proposals from Soviet and foreign readers---from college teachers and students, teachers of vocational and secondary schools, engineers and scientists. The author is grateful for all the comment and proposals.

There was a need therefore to revise the book completely. As in the first edition, the presentation of material has followed the aim of elucidating the physical nature of the phenomena discussed. But, where possible, the qualitative relations are also presented, often though without rigorous mathematics.

The manuscript was reviewed in detail by Prof. L. L. Dashkevich, Dr. of Technical Sciences, and Prof. I. G. Nekrashevich, Honored Scientist of the Belorussian Republic. It was perused by Prof. L. A. Gribov, Dr. of Mathematical and Physical Sciences, Assistant Prof. V. B. Zernov, and Z. S. Sazonova. The author extends sincere thanks for their efforts and criticism, which he took into account when revising the manuscript.

The author is also indebted to Senior Lecturer F. Zh. Vilf, Cand. of Technical Sciences, and Assistant Prof. Yu. A. Moma, Cand. of Technical Sciences, for manuals used in this textbook on superconductivity, Gunn effect, and principles of operation of impulse and high-frequency diodes, and to Z. I. Epifanova for all her work in preparing the manuscript.

The author will be most grateful for comment and proposals that might improve this book. They should be sent to the publishers.

\begin{flushright}
	\emph{G. I. E.}
\end{flushright}

%\pagestyle{mystyle}
