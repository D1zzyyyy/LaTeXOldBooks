% !TEX root = saveliev_physics_general_course_1.tex
%!TEX TS-program = pdflatex
%!TEX encoding = UTF-8 Unicode


\chapter{OSCILLATORY MOTION}\label{chap:7}

\section{General}\label{sec:7_1}

Oscillations are defined as processes distinguished by a certain degree of repetition. For example, the swings of a clock pendulum, the vibrations of a string or the leg of a tuning fork, and the voltage across the plates of a capacitor in a radio receiver circuit have this property of repetition.

Depending on the physical nature of the repeating process, we distinguish mechanical, electromagnetic, sound, and other oscillations. In the present chapter, we shall deal with mechanical oscillations.

Oscillations (vibrations) are widespread in nature and engineering. They often have a negative influence. The oscillations of a railway bridge due to the impacts imparted to it by the wheels of a train passing over the rail joints, the vibrations of a ship's hull caused by rotation of the propeller, the vibrations of the wings of an aircraft are all processes that may have catastrophic consequences. The task in such cases is to prevent the setting up of oscillations or at any rate to prevent them from reaching dangerous magnitudes.
Oscillatory processes are also at the very foundation of various branches of engineering. For instance, radio engineering owes its very existence to oscillatory processes. Depending on the nature of the action on an oscillating system, we distinguish free (or natural) oscillations, forced oscillations, auto-oscillations, and parametric oscillations.

\textbf{Free} or \textbf{natural oscillations} occur in a system left alone after an impetus was imparted to it or it was brought out of the equilibrium position. An example are the oscillations of a ball suspended on a string (a pendulum). To initiate oscillations, we may either push the ball or move it to a side and release it. 

In \textbf{forced oscillations}, the oscillating system is acted upon by an external periodically changing force. An example here are the oscillations of a bridge set up when people walking in step pass over it.

\textbf{Auto-oscillations}, like forced ones, are attended by the action of external forces on the oscillating system, but the moments of time when these actions are exerted are set by the oscillating system itself---the latter controls the external action. Examples of an auto-oscillating system are clocks in which a pendulum receives pushes at the expense of the energy of a lifted weight or a coiled spring, and these pushes occur when the pendulum passes through its middle position.

In \textbf{parametric oscillations}, external action causes periodic changes in a parameter of a system, for instance, in the length of a thread on which an oscillating ball is suspended.

\textbf{Harmonic oscillations} are the simplest ones. These are oscillations when the oscillating quantity (for example, the deflection of a pendulum) changes with time according to a sine or cosine law. This kind of oscillations is especially important for the following reasons: first, oscillations in nature and engineering are often close to harmonic ones in their character, and, second, periodic processes of a different form (with a different time dependence) can be represented as the superposition of several harmonic oscillations.

\section{Small-Amplitude Oscillations}\label{sec:7_2}

